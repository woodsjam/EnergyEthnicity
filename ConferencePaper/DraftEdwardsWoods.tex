\documentclass{article}
\usepackage{draftwatermark}
\usepackage{longtable}
\SetWatermarkText{DRAFT}
\SetWatermarkScale{1}

\author{Michael Edwards\\ 
  James Woods}
\title{Housing Market Institutions Drive Race and Ethnicity Differences in Energy Consumption}


\usepackage{Sweave}
\begin{document}
\maketitle
\Sconcordance{concordance:DraftEdwardsWoods.tex:DraftEdwardsWoods.Rnw:%
1 11 1 1 0 23 1 1 310 1 7 16 1 1 3 25 0 1 2 4 1 1 3 1 2 8 1 1 3 1 2 37 %
1 1 8 62 0 1 3 13 1 1 15 98 0 1 2 8 1 1 7 44 0 1 3 1 1 1 6 16 0 1 1 223 %
0 1 3 1 1 1 4 12 1 1 54 10 1}


\begin{abstract}

When socio-demographic factors are considered in any kind of analysis of household electric and gas utility data, it is common to observe differences in energy use between households with different self-reported race and ethnicity compositions. These differences persist controlling for structure type, e.g., single family dwelling, age and size of housing units, and, other common control variables. Without the information necessary to better explain these differences, they are commonly summarized simply as cultural differences. This paper demonstrates that these differences can be partially explained by differential sorting by structure and ownership, i.e., endogenizing housing choice and rental decisions. We will show that these differences in energy consumption may be because of housing market institutions and restrictions.
\end{abstract}

\section{Introduction}

*******Observed differences evident between different households can at least be partially explained by race and ethnicity composition.  That is, there 

0
\begin{itemize}

  \item Differences in energy use by race and ethnicity is frequently reported in the literature
  \item Much of the difference has to do with differences in income levels and education but not frequently modeled correctly.
  \item Even with proper race and ethnicity controls with income, some patterns persist.
  \item We assert that these differences are caused, at least in part, by housing discriminations, forcing people into housing types different than what they would prefer without discrimination; and renting rather than buying.
  \item We show this with a model that endogenizes structure type and tenure decision.
  \item emphasize that this is early work.
\end{itemize}

\cite{RBase}




  \subsection{Race and Ethnicity in Conditional Demand}
  
  \subsection{How Race and Ethnicity are Interpreted}

\section{RECS}

RECS data set is the 2009 publication.  12083 observations are included in the data, of which 11395 are included in the struture-tenure model.
\begin{itemize}
  \item Description on RECS data size number of observations sampling method etc.
  \item Scope of data  
  \item Weighting to account for stratifieed sampling
  \item We need more tables and graphics here.
\end{itemize}



